
%% Author: Peter Audier
%% This is Peter Audier's .tex file for his resume and can be used
%  as a template.



\documentclass[10pt,letter,roman]{moderncv}
% note: other options include different font sizes (10pt, 11pt and 12pt), paper sizes (a4paper, letterpaper, a5paper, legalpaper, executivepaper and landscape) and font families (sans and roman)

% modern themes
% note: style options include casual, classic, oldstyle and banking
\moderncvstyle{banking}

% note: color options include orange, black, blue, purple, red, grey and green
\moderncvcolor{orange}

% character encoding
\usepackage[utf8]{inputenc}

% Is used for listing tools used during projects
\usepackage{multicol}

% gets rid of all page numbers
\usepackage{nopageno}

% adjust the page margins
\usepackage[scale=0.85]{geometry}


% personal data
\name{\textsc{Peter}}{\textsc{Audier}}

\phone[mobile]{978-837-2713}

\email{pxa4084@rit.edu}

\homepage{github.com/PeterAudier}

\extrainfo{\homepagesymbol\httplink{linkedin.com/in/peter-audier}}


% =====    Resume Contents    ===============================================================
\begin{document}

\makecvtitle

\small{A third-year student studying software engineering at Rochester Institute of Technology. Previously worked with a local town government as a project lead for a small team focused on developing a web portal for use by electrical technicians. Available starting winter of 2018.}

\section{\textsc{Relevant Experience}}

\begin{itemize}

\item{\cventry{May 2016--August 2016}{Energy Intern}{Energize Andover}{Andover, MA}{}{}}
  \begin{itemize}
    \item Problem: The Town of Andover wanted to reduce energy waste in its infrastructure.
    \item Solution: Energize Andover was created to collaborate with the Andover Plant Department to tackle this issue.
    \item	My first step was to analyze the problem by discussing the requirements with technicians.
    \item	Next I designed a web portal for the technicians and janitors to view the circuit topology of town buildings utilizing UML.
    \item	Over the course of several months, I led and instructed a team of four high school students with little to no prior experience on how to follow the incremental development model during their internship.
    \item	Tools used for the project include:
    \begin{itemize}
    \begin{multicols}{3}
      \item Django
      \item Python
      \item GitHub
      \item HTML
      \item JavaScript
      \item PostgreSQL
      \item UML
      \item CSS
      \item BACpypes
    \end{multicols}
    \end{itemize}
  \end{itemize}

\item{\cventry{Fall 2016}{General Programmer and Requirements Manager}{HealthNet}{RIT}{}{}}
	\begin{itemize}
    	\item Problem: The professor assigned a project to deal with the issue of communication between Doctors and their patients.
        \item Solution: Create a web portal for hospitals to optimize day-to-day workflow and manage both their employees and patients.
		\item Each of the five students on our team took on a role relating to the software development lifecycle. I was elected requirements coordinator and kept the team’s focus steered towards the same goals as our “customer”
 		\item Throughout the semester we created two releases and accommodated when customer requirements changed.
		\item Tools gained over the course of the project include:
		\begin{itemize}
    	\begin{multicols}{3}
            \item Django
      		\item Python
     		\item Bootstrap
     		\item SQLite
     		\item SVN
     		\item JavaScript
        \end{multicols}
        \end{itemize}
  	\end{itemize}
\vspace{3pt}
\item{\cventry{June 2018--August 2018}{Assistant Instructor}{Circuit Lab}{Somerville,MA}{}{\vspace{3pt} Worked with children ages 6-12 teaching programming and circuit skills. Kept open and professional communication with parents to provide the best experience for their children.}}
\vspace{3pt}
\item{\cventry{May 2016--January 2017}{Data Entry Clerk}{MT Unirepair}{Wilmington,MA}{}{\vspace{3pt}Instructed other clerks on how to access and use the HP databases. Also worked in the printer repair shop helping to refurbish printers.}}
\vspace{3pt}
\item{\cventry{September 2018--today}{Assistant Manager}{Spirit Halloween}{Rochester,NY}{}{\vspace{3pt}Supervises and coaches Associates. Also ensures all store functions run smoothly when store manager is absent. Enforces all company policies and procedures including closing the store.}}

\end{itemize}

\section{\textsc{Education}}

\begin{itemize}
\item{\cventry{2016--2021}{BS}{Rochester Institute of Technology}{Rochester}{\textit{Software Engineering}}{}}
\end{itemize}

\vspace{2pt}

\section{\textsc{Skills and Interests}}

\begin{itemize}

\item \textbf{Programming Languages:} C, C\#, Python, Django, Java, SQL, JavaScript, HTML, LaTeX, Arduino, Basic, Ruby, C++, CSS.
\vspace{3pt}
\item \textbf{Source Control:} Git, Github, SVN.
\vspace{3pt}
\item \textbf{Leadership Experience:} President of Venture Crew 921 for two terms, member of North Reading Union Congregational Church Council for two terms, Treasurer for Beekeeping club at RIT for one term.
\vspace{3pt}
\item \textbf{Community Service:} Member of North Reading Union Congregational Church Environmental Stewardship Ministry for four years, member of Boy Scout Troop 750 and Venture Crew 921 for 8 years, Habitat for Humanity.

\end{itemize}

\end{document}